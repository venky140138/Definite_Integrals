\item 
\begin{align*}
f(x) = \begin{vmatrix}
\sec x & \cos x & \sec^{2}x + \cot x \cosec x \\ \cos^{2}x & \cos^{2}x & \cosec^{2}x \\ 1 & \cos^{2}x & \cos^{2}x
\end{vmatrix}
\end{align*}
Then $\int_{0}^{\pi/2}f(x)dx$=......

\item The integral $\int_{0}^{1.5}[x^{2}]dx$, Where [ ] denotes the greatest integer function, equals.......

\item The value of 
\begin{align*}
\int_{-2}^{2}|1 - x^{2}|dx = 
\end{align*}

\item The value of 
\begin{align*}
\int_{\pi/4}^{3\pi/4}\frac{\phi}{1 + \sin\phi}d\phi = 
\end{align*}

\item The value of 
\begin{align*}
\int_{2}^{3}\frac{\sqrt{x}}{\sqrt{5 - x} + \sqrt{x}}dx = 
\end{align*}

\item If for non-zero x, 
\begin{align*}
af(x) + bf\left(\frac{1}{x}\right) = \frac{1}{x} - 5
\end{align*}
where $a \neq b$, then $\int_{1}^{2}f(x)dx$ = ...............

\item For $n > 0$, 
\begin{align*}
\int_{0}^{2\pi}\frac{x\sin^{2n}x}{\sin^{2n}x + \cos^{2n}x}dx = 
\end{align*}

\item The value of 
\begin{align*}
\int_{1}^{e^{37}}\frac{\pi \sin(\pi lnx)}{x}dx = 
\end{align*}

\item Let $\frac{d}{dx}F(x) = \frac{e^{\sin x}}{x}$, $x > 0$. If
\begin{align*}
\int_{1}^{4} \frac{2e^{\sin x^{2}}}{x} = F(k) - F(1)
\end{align*}
then one of the possible values of k is..........

\textbf{ True/False}

\item The value of the integral 
\begin{align*}
\int_{0}^{2a}\frac{f(x)}{f(x) + f(2a - x)}dx
\end{align*}
is equal to a.

\textbf{ MCQs with One Correct Answer}

\item The value of the definite integral $\int_{0}^{1}(1 + e^{-x^{2}})dx$ is
\begin{enumerate}
\item -1
\item 2
\item $1 + e^{-1}$
\item None of these
\end{enumerate} 

\item Let a, b, c be non-zero real numbers such that
\begin{align*}
\int_{0}^{1}(1+\cos^{8}x)(ax^2+bx+c)dx\\=\int_{0}^{2}(1+\cos^{8}x)(ax^2+bx+c)dx
\end{align*}
Then the quadratic equation $ax^2 + bx + c = 0$ has
\begin{enumerate}
\item no root in (0, 2)
\item at least one root in (0, 2)
\item a double root in (0, 2)
\item two imaginary roots
\end{enumerate}

\item The area bounded by the curves $y = f(x)$, the x-axis and the ordinates x = 1 and x = b is $(b - 1)\sin(3b + 4)$. Then $f(x)$ is
\begin{enumerate}
\item $(x - 1)\cos(3x + 4)$
\item $\sin(3x + 4)$
\item $\sin(3x + 4) + 3(x - 1)\cos(3x + 4)$
\item None of these
\end{enumerate}

\item The value of the integral
\begin{align*}
\int_{0}^{\pi/2}\frac{\sqrt{\cot x}}{\sqrt{\cot x} + \sqrt{\tan x}}dx = 
\end{align*}
\begin{enumerate}
\item $\pi/4$
\item $\pi/2$
\item $\pi$
\item None of these
\end{enumerate}

\item For any integer n the integral
\begin{align*}
\int_{0}^{\pi}e^{\cos^{2}x}\cos^{3}(2n + 1)dx
\end{align*}
has the value
\begin{enumerate}
\item $\pi$
\item 1
\item 0
\item None of these
\end{enumerate}

\item Let $f: R \to R$ and $g: R \to R$ be continuous functions. Then the value of the integral
\begin{align*}
\int_{-\pi/2}^{\pi/2}[f(x) + f(-x)][g(x) - g(-x)]dx = 
\end{align*}
\begin{enumerate}
\item $\pi$
\item 1
\item -1
\item 0
\end{enumerate}

\item The value of 
\begin{align*}
\int_{0}^{\pi/2}\frac{dx}{1 + \tan^{3}x} = 
\end{align*}
\begin{enumerate}
\item 0
\item 1
\item $\pi/2$
\item $\pi/4$
\end{enumerate}

\item If $f(x) = A\sin(\frac{\pi x}{2}) + B$, $f'(\frac{1}{2}) = \sqrt{2}$ and $\int_{0}^{1}f(x)dx = \frac{2A}{\pi}$, then constants A and B are
\begin{enumerate}
\item $\frac{\pi}{2}$ and $\frac{\pi}{2}$
\item $\frac{2}{\pi}$ and $\frac{3}{\pi}$
\item 0 and $\frac{-4}{\pi}$
\item $\frac{4}{\pi}$ and 0
\end{enumerate}

\item The value of 
$\int_{\pi}^{2\pi}[2\sin x]dx$
where [ ] represents the greatest integer funcion is
\begin{enumerate}
\item $\frac{-5\pi}{3}$
\item $-\pi$
\item $\frac{5\pi}{3}$
\item $-2\pi$
\end{enumerate}

\item If 
\begin{align*}
g(x) = \int_{0}^{x}\cos^{4}t dt
\end{align*}
then $g(x + \pi)$ equals
\begin{enumerate}
\item $g(x) + g(\pi)$
\item $g(x) - g(\pi)$
\item $g(x)g(\pi)$
\item $\frac{g(x)}{g(\pi)}$
\end{enumerate}

\item
\begin{align*}
\int_{\pi/4}^{3\pi/4}\frac{dx}{1 + \cos x} = 
\end{align*}
\begin{enumerate}
\item 2
\item -2
\item 1/2
\item -1/2
\end{enumerate}

\item If for a real number y, [y] is the greatest integer less than or equal to y, then the value of the integral
\begin{align*}
\int_{\pi/2}^{3\pi/2}[2\sin x]dx = 
\end{align*}
\begin{enumerate}
\item -$\pi$
\item 0
\item $\pi/2$
\item $\pi/2$
\end{enumerate}

\item Let
\begin{align*}
g(x) = \int_{0}^{x}f(t) dt
\end{align*}
where f is such that $\frac{1}{2} \leq f(t) \leq 1$ for $t \in [0, 1]$ and $0 \leq f(t) \leq \frac{1}{2}$, for $t \in [1, 2]$. Then g(2) satisfies the inequality
\begin{enumerate}
\item $\frac{3}{2} \leq g(2) < \frac{1}{2}$
\item $0 \leq g(2) < 2$
\item $\frac{3}{2} < g(2) \leq \frac{5}{2}$
\item $2 < g(2) < 4$
\end{enumerate}

\item If 
\begin{align*}
f(x) = 
\left\lbrace
\begin{array}{ll}
      e^{\cos x}\sin x & for |x| \leq 2 \\
      2 & otherwise \\
\end{array} 
\right\rbrace
\end{align*} 
then $\int_{-2}^{3}f(x)dx$ = 
\begin{enumerate}
\item 0
\item 1
\item 2
\item 3
\end{enumerate}

\item The value of the integral 
\begin{align*}
\int_{e^{-1}}^{e^{2}}\begin{vmatrix} \frac{log_ex}{x} \end{vmatrix}dx =
\end{align*}
\begin{enumerate}
\item 3/2
\item 5/2
\item 3
\item 5
\end{enumerate}

\item The value of 
\begin{align*}
\int_{-\pi}^{\pi}\frac{\cos^{2}x}{1 + a^{x}}dx = 
\end{align*}
\begin{enumerate}
\item $\pi$
\item $a\pi$
\item $\pi/2$
\item $2\pi$
\end{enumerate}

\item The area bounded by the curves $y = |x| - 1$ and $y = -|x| + 1$ is
\begin{enumerate}
\item 1
\item 2
\item $2\sqrt{2}$
\item 4
\end{enumerate}

\item Let
\begin{align*}
f(x) = \int_{1}^{x}\sqrt{2 - t^{2}}dt
\end{align*}
Then the real roots of the equation $x^2 - f'(x) = 0$ are
\begin{enumerate}
\item $\pm 1$
\item $\pm \frac{1}{\sqrt{2}}$
\item $\pm \frac{1}{2}$
\item 0 and 1
\end{enumerate}

\item Let  $T > 0$ be a fixed real number. Suppose f is a continuous function such that for all $x \in R$, $f(x + T) = f(x)$. If $I = \int_{0}^{T}f(x)dx$ then the value of $\int_{3}^{3 + 3T}f(2x)dx$ is
\begin{enumerate}
\item 3/2I
\item 2I
\item 3I
\item 6I
\end{enumerate}

\item The integral
\begin{align*}
\int_{-1/2}^{1/2}([x] + ln(\frac{1 + x}{1 - x}))dx = 
\end{align*}
\begin{enumerate}
\item $\frac{-1}{2}$
\item 0
\item 1
\item $2ln(\frac{1}{2})$
\end{enumerate}

\item If 
$l(m, n) = \int_{0}^{1}t^{m}(1 + t)^{n}dt$
then the expression for l(m, n) in terms of l(m + 1, n - 1) is
\begin{enumerate}
\item $\frac{2^n}{m + 1} - \frac{n}{m + 1}l(m + 1, n -1)$
\item $\frac{n}{m + 1}l(m + 1, n -1)$
\item $\frac{2^n}{m + 1} + \frac{n}{m + 1}l(m + 1, n -1)$
\item $\frac{m}{m + 1}l(m + 1, n -1)$
\end{enumerate}

\item If 
\begin{align*}
f(x) = \int_{x^{2}}^{x^{2} + 1}e^{-t^{2}}dt
\end{align*}
then, f(x) increases in
\begin{enumerate}
\item (-2, 2)
\item No value of x
\item $(0, \infty)$
\item $(-\infty, 0)$
\end{enumerate}

\item The area bounded by the curves $y = \sqrt{x}$, 2y + 3 = x and x-axis in the $1^{st}$ quadrant is
\begin{enumerate}
\item 9
\item 27/4
\item 36
\item 18
\end{enumerate}

\item If f(x) is differentiable and 
\begin{align*}
\int_{0}^{t^{2}}xf(x)dx = \frac{2}{5}t^{5}
\end{align*}
then $f(\frac{4}{25})$ equals
\begin{enumerate}
\item 2/5
\item -5/2
\item 1
\item 5/2
\end{enumerate}

\item The value of the integral
\begin{align*}
\int_{0}^{1}\sqrt{\frac{1 - x}{1 + x}}dx = 
\end{align*}
\begin{enumerate}
\item $\frac{\pi}{2} + 1$
\item $\frac{\pi}{2} - 1$
\item -1
\item 1
\end{enumerate}

\item The area enclosed between the curves $y = ax^{2}$  and $x = ay^{2}(a > 0)$ is 1 sq. unit, then the value of a is
\begin{enumerate}
\item $1/\sqrt{3}$
\item 1/2
\item 1
\item 1/3
\end{enumerate}

\item 
\begin{align*}
\int_{-2}^{0}\{x^3+3x^2+3x+3+(x+1)\cos(x+1)\}dx = 
\end{align*}
\begin{enumerate}
\item -4
\item 0
\item 4
\item 6
\end{enumerate}

\item The area bounded by the parabolas $y = (x + 1)^{2}$ and $y = (x - 1)^{2}$ and the line $y = 1/4$ is
\begin{enumerate}
\item 4 sq.units
\item 1/6 sq.units
\item 4/3 sq.units
\item 1/3 sq.units
\end{enumerate}

\item The area of the region between the curves $y = \sqrt{\frac{1 + \sin x}{\cos x}}$ and $y = \sqrt{\frac{1 - \sin x}{\cos x}}$ bounded by the lines x = 0 and $x = \frac{\pi}{4}$ is
\begin{enumerate}
\item $\int_{0}^{\sqrt{2} - 1}\frac{t}{(1 + t^2)\sqrt{1 - t^2}}dt$
\item $\int_{0}^{\sqrt{2} - 1}\frac{4t}{(1 + t^2)\sqrt{1 - t^2}}dt$
\item $\int_{0}^{\sqrt{2} + 1}\frac{4t}{(1 + t^2)\sqrt{1 - t^2}}dt$
\item $\int_{0}^{\sqrt{2} + 1}\frac{t}{(1 + t^2)\sqrt{1 - t^2}}dt$
\end{enumerate}

\item Let f be a non-negative function defined on the interval [0, 1]. If
\begin{align*}
\int_{0}^{x}\sqrt{1 - (f'(t))^2}dt = \int_{0}^{x}f(t)dt,
\end{align*}
$0 \leq x \leq 1$, and f(0) = 0, then
\begin{enumerate}
\item $f(\frac{1}{2}) < \frac{1}{2}$ and $f(\frac{1}{3}) > \frac{1}{3}$
\item $f(\frac{1}{2}) > \frac{1}{2}$ and $f(\frac{1}{3}) > \frac{1}{3}$
\item $f(\frac{1}{2}) < \frac{1}{2}$ and $f(\frac{1}{3}) < \frac{1}{3}$
\item $f(\frac{1}{2}) > \frac{1}{2}$ and $f(\frac{1}{3}) < \frac{1}{3}$
\end{enumerate}

\item The value of 
\begin{align*}
\lim_{x \to 0}\frac{1}{x^3}\int_{0}^{x}\frac{tln(1 + t)}{t^4 + 4}dt
\end{align*}
\begin{enumerate}
\item 0
\item $\frac{1}{12}$
\item $\frac{1}{24}$
\item $\frac{1}{64}$
\end{enumerate}

\item Let f be a real-valued function defined on the interval (-1, 1) such that 
\begin{align*}
e^{-x}f(x) = 2 + \int_{0}^{x}\sqrt{t^4 + 1}dt
\end{align*}
for all $x \in (-1, 1)$, and let $f^{-1}$ be the inverse function of f. Then $(f^{-1})'(2)$ is equal to
\begin{enumerate}
\item 1
\item 1/3
\item 1/2
\item 1/e
\end{enumerate}

\item The value of 
\begin{align*}
\int_{\sqrt\ln2}^{\sqrt{ln3}}\frac{x\sin x^2}{\sin x^2 + \sin(ln6 - x^2)}dx = 
\end{align*}
\begin{enumerate}
\item $\frac{1}{4}ln\frac{3}{2}$
\item $\frac{1}{2}ln\frac{3}{2}$
\item $ln\frac{3}{2}$
\item $\frac{1}{6}ln\frac{3}{2}$
\end{enumerate}

\item Let the straight line x = b divide the area enclosed by $y = (1 - x)^2$, y = 0 and x = 0 into two parts $R_1(0 \leq x \leq b)$ and $R_2(b \leq x \leq 1)$ such that $R_1 - R_2  = \frac{1}{4}$. Then b equlas
\begin{enumerate}
\item 3/4
\item 1/2
\item 1/3
\item 1/4
\end{enumerate}

\item Let $f: [-1, 2] \to [0, \infty)$ be a continuous function such that $f(x) = f(1 - x)$ for all $x \in [-1, 2]$. Let 
$R_1 = \int_{-1}^{2}xf(x)dx$, x = -1, x = 2 and the x-axis. Then
\begin{enumerate}
\item $R_1 = 2R_2$
\item $R_1 = 3R_2$
\item $2R_1 = R_2$
\item $3R_1 = R_2$
\end{enumerate}

\item The value of the integral
\begin{align*}
\int_{-\pi/2}^{\pi/2}(x^2 + ln\frac{\pi + x}{\pi - x})\cos x dx
\end{align*}
\begin{enumerate}
\item 0
\item $\frac{\pi^2}{2} - 4$
\item $\frac{\pi^2}{4} + 4$
\item $\frac{\pi^2}{2}$
\end{enumerate}

\item The area enclosed by the curves $y = \sin x + \cos x$ and $y = |\cos x - \sin x|$ over the interval $[0, \frac{\pi}{2}]$ is
\begin{enumerate}
\item $4(\sqrt{2} - 1)$
\item $2\sqrt{2}(\sqrt{2} - 1)$
\item $2(\sqrt{2} + 1)$
\item $2\sqrt{2}(\sqrt{2} + 1)$
\end{enumerate}

\item Let $f: [\frac{1}{2}, 1] \to R$(the set of all real number) be a positive, non-constant and diffrentiable function such that $f'(x) < 2f(x)$ and $f(\frac{1}{2}) = 1$. Then the value of $\int_{1/2}^{1}f(x)dx$ lies in the interval
\begin{enumerate}
\item (2e-1, 2e)
\item (e - 1, 2e - 1)
\item $(\frac{e - 1}{2}, e - 1)$
\item $(0, \frac{e - 1}{2})$
\end{enumerate}

\item The following integral
\begin{align*}
\int_{\pi/4}^{\pi/2}(2\cosec x)^{17}dx
\end{align*}
is equal to
\begin{enumerate}
\item $\int_{0}^{log(1 + \sqrt{2})}2(e^u + e^{-u})^{16}dx$
\item $\int_{0}^{log(1 + \sqrt{2})}(e^u + e^{-u})^{17}dx$
\item $\int_{0}^{log(1 + \sqrt{2})}2(e^u - e^{-u})^{17}dx$
\item $\int_{0}^{log(1 + \sqrt{2})}2(e^u - e^{-u})^{16}dx$
\end{enumerate}

\item The value of 
\begin{align*}
\int_{\pi/2}^{\pi/2}\frac{x^2 \cos x}{1 + e^x}dx
\end{align*}
is equal to
\begin{enumerate}
\item $\frac{\pi^2}{4} - 2$
\item $\frac{\pi^2}{4} + 2$
\item $\pi^2 - e^{\frac{\pi}{2}}$
\item $\pi^2 + e^{\frac{\pi}{2}}$
\end{enumerate}

\item Area of the region
\begin{align*}
\{(x, y) \in R^2: y \geq \sqrt{|x + 3|}, 5y \leq x + 9 \leq 15\}
\end{align*}
is equal to
\begin{enumerate}
\item $\frac{1}{6}$
\item $\frac{4}{3}$
\item $\frac{3}{2}$
\item $\frac{5}{3}$
\end{enumerate}

\item The area of the region
\begin{align*}
\{(x, y): xy \leq 8, 1 \leq y \leq x^2\} = 
\end{align*}
\begin{enumerate}
\item $8log_e2 - \frac{14}{3}$
\item $16log_e2 - \frac{14}{3}$
\item $8log_e2 - \frac{7}{3}$
\item $16log_e2 - 6$
\end{enumerate}

\textbf{MCQs with One or More than One Correct Answer}

\item If 
\begin{align*}
\int_{0}^{x}f(t)dt = x + \int_{x}^{t} tf(t)dt
\end{align*}
then the value of f(1) is
\begin{enumerate}
\item 1/2
\item 0
\item 1
\item -1/2
\end{enumerate}

\item Let $f(x) = x - [x]$, for every real number x, where [x] is the integral part of x. Then 
\begin{align*}
\int_{-1}^{1}f(x)dx = 
\end{align*}
\begin{enumerate}
\item 1
\item 2
\item 0
\item 1/2
\end{enumerate}

\item For which of the following values of m, is the area of the region bounded by the curve $y = x - x^{2}$ and the line 
$y = mx$ equals 9/2?
\begin{enumerate}
\item -4
\item -2
\item  2 
\item  4
\end{enumerate}

\item Let $f(x)$ be a non-constant twice diffrentiable function defined on $(-\infty, \infty)$ such that $f(x) = f(1 - x)$ and $f'(\frac{1}{4}) = 0$. Then,
\begin{enumerate}
\item $f''(x)$ vanishes at least twice on [0, 1]
\item $f'(\frac{1}{2}) = 0$
\item $\int_{-1/2}^{1/2}f(x + \frac{1}{2})\sin x dx = 0$
\item $\int_{0}^{1/2}f(t)e^{\sin \pi t}dt = \int_{1/2}^{1}f(1 - t)e^{\sin \pi t}dt$
\end{enumerate}

\item Area of the region bounded by the curve $y = e^x$ and lines x = 0 and y = e is
\begin{enumerate}
\item e - 1
\item $\int_{1}^{e}ln(e + 1 - y)dy$
\item $e - \int_{0}^{1}e^xdx$
\item $\int_{1}^{e}ln ydy$
\end{enumerate}

\item If 
\begin{align*}
I_n = \int_{-\pi}^{\pi}\frac{\sin nx}{(1 + \pi^{x})\sin x}dx
\end{align*}
where n = 0, 1 , 2..... then
\begin{enumerate}
\item $I_n = I_{n + 2}$
\item $\sum_{m = 1}^{10}I_{2m + 1} = 10\pi$
\item $\sum_{m = 1}^{10}I_{2m} = 0$
\item $I_n = I_{n + 1}$
\end{enumerate} 

\item The value(s) of
\begin{align*}
\int_{0}^{1}\frac{x^4(1 - x)^4}{1 + x^2}dx = 
\end{align*}
\begin{enumerate}
\item $\frac{22}{7} - \pi$
\item $\frac{2}{105}$
\item $0$
\item $\frac{71}{15} - \frac{3\pi}{2}$
\end{enumerate}

\item Let f be a real-valued function defined on the interval $(0, \infty)$ by 
\begin{align*}
f(x) = lnx + \int_{0}^{x}\sqrt{1 + \sin t}dt
\end{align*}
Then which of the following statement(s) is(are) true?
\begin{enumerate}
\item $f''(x)$ exists for all $x \in (0, \infty)$
\item $f'(x)$ exists for all $x \in (0, \infty)$ and $f'$ is continuous on $(0, \infty)$, but not differentiable on $(0, \infty)$
\item There exists $\alpha > 1$ such that $|f'(x)| < |f(x)|$ for all $x \in (\alpha, \infty)$
\item There exists $\beta > 1$ such that $|f'(x)| + |f(x)| \leq \beta$ for all $x \in (0, \infty)$
\end{enumerate}

\item Let S be the area of the region enclosed by $y = e^{x^{2}}$, y = 0, x = 0 and x = 1: then
\begin{enumerate}
\item $S \geq \frac{1}{e}$
\item $S \geq 1-\frac{1}{e}$
\item $S \leq \frac{1}{4}(1 + \frac{1}{\sqrt{e}})$
\item $S \leq \frac{1}{\sqrt{2}} + \frac{1}{\sqrt{e}}(1 - \frac{1}{\sqrt{2}})$
\end{enumerate}

\item The option(s) with the values of a and L that satisfy the following equation is(are)
\begin{align*}
\frac{\int_{0}^{4\pi}e^{t}(\sin^{6}at + \cos^{4}at)dt}{\int_{0}^{\pi}e^{t}(\sin^{6}at + \cos^{4}at)dt} = L?
\end{align*}
\begin{enumerate}
\item a = 2, $L = \frac{e^{4\pi} - 1}{e^{\pi} - 1}$
\item a = 2, $L = \frac{e^{4\pi} + 1}{e^{\pi} + 1}$
\item a = 4, $L = \frac{e^{4\pi} - 1}{e^{\pi} - 1}$
\item a = 4, $L = \frac{e^{4\pi} + 1}{e^{\pi} + 1}$
\end{enumerate}

\item Let 
\begin{align*}
f(x) = 7\tan^{8}x + 7\tan^{6}x - 3\tan^{4}x - 3\tan^{2}x
\end{align*}
for all $x \in (\frac{\pi}{2}, \frac{\pi}{2})$. Then the correct expression(s) is(are)
\begin{enumerate}
\item $\int_{0}^{\pi/4}xf(x)dx = \frac{1}{12}$
\item $\int_{0}^{\pi/4}f(x)dx = 0$
\item $\int_{0}^{\pi/4}xf(x)dx = \frac{1}{6}$
\item $\int_{0}^{\pi/4}f(x)dx = 1$
\end{enumerate}

\item Let $f'(x) = \frac{192x^{3}}{2 + \sin^{4}\pi x}$ for all $x \in R$ with $f(\frac{1}{2}) = 0$. If 
\begin{align*}
m < \leq \int_{1/2}^{1}f(x)dx \leq M
\end{align*}
then the possible values of m and M are
\begin{enumerate}
\item m = 13, M = 24
\item m = 0.25, M = 0.5
\item m = -11, M = 0
\item m = 1, M = 12
\end{enumerate}

\item Let
\begin{align*}
f(x) = \lim_{x \to \infty}(\frac{n^n(x+n)(x+\frac{n}{2}).....(x+\frac{n}{n})}{n!(x^2+n^2)(x^2+\frac{n^2}{4}).......(x^2+\frac{n^2}{n^2})})^{\frac{x}{n}} 
\end{align*}
for all $x > 0$. Then
\begin{enumerate}
\item $f(\frac{1}{2}) \geq f(1)$
\item $f(\frac{1}{3}) \leq f(\frac{2}{3})$
\item $f'(2) \leq 0$
\item $\frac{f'(3)}{f(3)} \geq \frac{f'(2)}{f(2)}$
\end{enumerate}

\item Let $f: R \to (0, 1)$ be a continuous function. Then, which of the following function(s) has(have) the value zero at some point in the interval (0, 1)?
\begin{enumerate}
\item $x^9 - f(x)$
\item $x - \int_{0}^{\frac{\pi}{2} - x}f(t)\cos t dt$
\item $e^{x} - \int_{0}^{x}f(t)\sin t dt$
\item $f(x) + \int_{0}^{\pi/2}f(t)\sin t dt$
\end{enumerate}

\item If 
\begin{align*}
g(x) = \int{\sin x}^{\sin (2x)}\sin^{-1}(t)dt
\end{align*}
then,
\begin{enumerate}
\item $g'(\frac{\pi}{2}) = -2\pi$
\item $g'(\frac{-\pi}{2}) = 2\pi$
\item $g'(\frac{\pi}{2}) = 2\pi$
\item $g'(\frac{-\pi}{2}) = -2\pi$
\end{enumerate}

\item If the line $sx = \alpha$ divides the area of the region
\begin{align*}
R = \{(x, y) \in R^{2}: x^3 \leq y \leq x, 0 \leq x \leq 1\}
\end{align*}
into two equal parts, then
\begin{enumerate}
\item $0 < \alpha \leq \frac{1}{2}$
\item $\frac{1}{2} < \alpha < 1$
\item $2\alpha^{4} - 4\alpha^{2} + 1 = 0$
\item $\alpha^{4} + 4\alpha^{2} - 1 = 0$
\end{enumerate}

\item If 
\begin{align*}
I = \sum_{k = 1}^{98}\int_{k}^{k + 1}\frac{k + 1}{x(x + 1)}dx, then
\end{align*}
\begin{enumerate}
\item $1 > log_e99$
\item $1 < log_e99$
\item $1 < \frac{49}{50}$
\item $1 > \frac{49}{50}$
\end{enumerate}

\item For, $a \in R$, $|a| > 1$, let
\begin{align*}
\lim_{x \to \infty}\frac{1 + 2^{3/2} + .......+ n^{3/2}}{n^{7/3}(\frac{1}{(an + 1)^2} + \frac{1}{(an + 2)^2}+.......+\frac{1}{(an + n)^2})} = 54
\end{align*}
Then the possible value(s) of a is/are
\begin{enumerate}
\item -9
\item 7
\item 6
\item 8
\end{enumerate}

\textbf{Subjective Problems}

\item Find the area bounded by the curve $x^2 = 4y$ and the straight line x = 4y - 2.

\item Show that:
\begin{align*}
\lim_{n \to \infty}(\frac{1}{n + 1} + \frac{1}{n + 2}+......+\frac{1}{6n}) = log 6
\end{align*}

\item Show that
\begin{align*}
\int_{0}^{\pi}xf(\sin x)dx = \frac{\pi}{2}\int_{0}^{\pi}f(\sin x)dx
\end{align*}

\item Find the value of
\begin{align*}
\int_{-1}^{3/2}|x \sin \pi x|dx
\end{align*}

\item For any real t, $x = \frac{e^t + e^{-t}}{2}$, $y = \frac{e^t - e^{-t}}{2}$ is a point on the hyperbola $x^2 - y^2 = 1$. Show that the area bounded by the this hyperbola and the line joining its centre to the points corresponding to $t_1$ and $-t_1$ is $t_1$.

\item Evaluate
\begin{align*}
\int_{0}^{\pi/4}\frac{\sin x + \cos x}{9 + 16\sin 2x}dx
\end{align*}

\item Find the area bounded by the x-axis, part of the curve $y = (1 + \frac{8}{x^2})$ and the ordinates at x = 2 and x = 4. If the ordinate at x = a divides the area into two equal parts, find a.

\item Evaluate the follwing
\begin{align*}
\int_{0}^{1/2}\frac{x\sin^{-1}x}{\sqrt{1 - x^2}}dx.
\end{align*}

\item Find the area of the region bounded by the x-axis and the curves defined by
\begin{align*}
y = \tan x, \frac{-\pi}{3} \leq x \leq \frac{\pi}{3}
\end{align*}
\begin{align*}
y = \cot x, \frac{\pi}{6} \leq x \leq \frac{3\pi}{2}
\end{align*}

\item Given a function $f(x)$ such that
\begin{enumerate}
\item it is integratable over every interval on the real line and
\item $f(t + x) = f(x)$, for every x and a real t, then Show that the integral $\int_{a}^{a + 1}f(x)dx$ is independent of a.
\end{enumerate}

\item Evaluate the following:
\begin{align*}
\int_{0}^{\pi/2}\frac{x\sin x \cos x}{\cos^{4}x + \sin^{4}x}dx
\end{align*}

\item Sketch the region bounded by the curves $y = \sqrt{5 - x^{2}}$ and $y = |x - 1|$ and find its area.

\item Evaluate
\begin{align*}
\int_{0}^{\pi}\frac{x dx}{1 + \cos \alpha \sin x}, 0 < \alpha < \pi
\end{align*}

\item Find the area bounded by the curves 
\begin{align}
x^3 + y^{2} = 25,
\end{align}
$4y = |4 - x^{2}|$ and x = 0 above the x-axis.

\item Find the area of the region bounded by the curves $C: y = \tan x$, tangent drawn to C at $x = \frac{\pi}{4}$ and the x-axis.

\item Evaluate
\begin{align*}
\int_{0}^{1}log[\sqrt{1 - x} + \sqrt{1 + x}]dx
\end{align*}

\item If $f$ and $g$ are continuous function on [0, a] satisfying $f(x) = f(a - x)$ and $g(x) + g(a - x) = 2$, then Show that 
\begin{align*}
\int_{0}^{a}f(x)g(x)dx = \int_{0}^{a}f(x)dx
\end{align*}

\item Show that
\begin{align*}
\int_{0}^{\pi/2}f(\sin 2x)\sin x dx = \sqrt{2}\int_{0}^{\pi/4}f(\cos 2x)\cos x dx.
\end{align*}

\item Prove that for any positive integer k,
\begin{align*}
\frac{\sin 2kx}{\sin x} = 2[\cos x + \cos 3x + .........+\cos(2k - 1)x]
\end{align*}
Hence prove that
\begin{align*}
\int_{0}^{\pi/2}\sin 2kx \cot x dx = \frac{\pi}{2}
\end{align*}

\item Compute the area of the region bounded by the curves $y = ex lnx$ and $y = \frac{lnx}{ex}$ where $lne = 1$.

\item Sketch the curves and identify the region bounded by the $x = \frac{1}{2}$, x = 2, $y = lnx$ and $y = 2^x$. Find the area of the region.  

\item Evaluate
\begin{align*}
\int_{0}^{\pi}\frac{x \sin 2x \sin(\frac{\pi}{2}\cos x)}{2x - \pi}dx
\end{align*}

\item Sketch the region bounded by the curves $y = x^2$ and $y = \frac{2}{1 + x^2}$. Find the area.

\item Determine a positive integer $n \leq 5$, such that
\begin{align*}
\int_{0}^{1}e^x(x - 1)^{n}dx = 16 - 6e
\end{align*}

\item Evaluate
\begin{align*}
\int_{2}^{3}\frac{2x^5 + x^4 - 2x^3 + 2x^2 + 1}{(x^2 + 1)(x^4 - 1)}dx
\end{align*}

\item Show that
\begin{align*}
\int_{0}^{n\pi + v}|\sin x|dx = 2n + 1 - \cos v
\end{align*}
where n is a positive integer and $0 \leq v < \pi$.

\item In what ratio does the x-axis divide the area of the region bounded by the parabolas $y = 4x - x^2$ and $y = x^2 - x$?

\item Let
\begin{align*}
I_m = \int_{0}^{\pi}\frac{1 - \cos mx}{1 - \cos x}dx
\end{align*}
Use the mathematical induction to prove that $I_m = m\pi$, m = 0, 1, 2........

\item Evaluate the definite integral
\begin{align*}
\int_{-1/\sqrt{3}}^{1/\sqrt{3}}\left(\frac{x^4}{1 - x^4}\right)\cos^{-1}\left(\frac{2x}{1 + x^2}\right) dx
\end{align*}

\item Consider a square with vertices at (1, 1), (-1, 1), (-1, -1) and (1, -1). Let S be the region consisting of all points inside the square which are near to the origin than to any edge. Sketch the region S and find its area.

\item Let $A_n$ be the area bounded by the curve $y = (\tan x)^n$ and the lines x = 0, y = 0 and $x = \frac{\pi}{4}$. Prove that $n > 2$,
\begin{align*}
A_n + A_{n-2} = \frac{1}{n - 1}
\end{align*}
and deduce
\begin{align*}
\frac{1}{2n + 2} < A_n < \frac{1}{2n - 2}
\end{align*}

\item Determine the value of
\begin{align*}
\int_{-\pi}^{\pi}\frac{2x(1 + \sin x)}{1 + \cos^{2}x}dx.
\end{align*}

\item Let 
\begin{align*}
f(x) = Maximum\{x^2, (1 - x^2), 2x(1 - x)\}
\end{align*}
where $0 \leq x \leq 1$. Determine the area of the region bounded by the curves $y = f(x)$ x-axis x = 0 and x = 1.

\item Prove that
\begin{align*}
\int_{0}^{1}\tan^{-1}\left(\frac{1}{1 - x + x^2}\right) dx = 2\int_{0}^{1}\tan^{-1}xdx
\end{align*}
Hence or otherwise, evaluate the integral
\begin{align*}
\int_{0}^{1}\tan^{-1}\left(1 - x + x^{2}\right) dx
\end{align*}

\item Integrate
\begin{align*}
\int_{0}^{\pi}\frac{e^{\cos x}}{e^{\cos x} + e^{-\cos x}}dx.
\end{align*}

\item Let $f(x)$ be a continuous function given by
\begin{align*}
f(x) = 
\left\lbrace
\begin{array}{ll}
      2x & |x| \leq 1 \\
      x^2 + ax + b & |x| > 1 \\
\end{array} 
\right\rbrace
\end{align*}
Find the area of the region in the third quadrant bounded by the curves $x = -2y^{2}$ and $y = f(x)$ lying on the left of the line $8x + 1 = 0$.

\item For $x > 0$, let 
\begin{align*}
f(x) = \int_{e}^{x}\frac{ln t}{1 + t}dt
\end{align*}
Find the function $f(x) + f(\frac{1}{x})$ and show that $f(e) + f(\frac{1}{e}) = \frac{1}{2}$.

\item Let $b \neq 0$ and for j = 0, 1, 2, ......n, let $S_j$ be the area of the region bounded by the y-axis and the curve $xe^{ay} = \sin by$, $\frac{jr}{b} \leq y \leq \frac{(j + 1)\pi}{b}$. Show that $S_0$, $S_1$, $S_2$,.......$S_n$ are in geometric progression. Also, find their sum for a = -1 and $b = \pi$.

\item Find the area of the region bounded by the curves $y = x^2$, $y = |2 - x^2|$ and y = 2, which lies to the right of the line x = 1.

\item If f is an even function then prove that
\begin{align*}
\int_{0}^{\pi/2}f(\cos 2x)\cos x dx = \sqrt{2}\int_{0}^{\pi/4}f(\sin 2x)\cos x dx
\end{align*}

\item Find the value of
\begin{align*}
\int_{-\pi/3}^{\pi/3}\frac{\pi + 4x^{3}}{2 - \cos \left(|x| + \frac{\pi}{3}\right)}dx
\end{align*}

\item If
\begin{align*}
y(x) = \int_{\pi^{2}/16}^{x^2}\frac{\cos x \cos\sqrt{\theta}}{1 + \sin^{2}\sqrt{\theta}}d\theta
\end{align*}
find $\frac{dy}{dx}$ at $x = \pi$

\item Evaluate
\begin{align*}
\int_{0}^{\pi}e^{|\cos x|}\left(2\sin\left(\frac{1}{2}\cos x\right) + 3\cos\left(\frac{1}{2}\cos x\right)\right)\sin x dx
\end{align*}

\item Find the area bounded by the curves $x^2 = y$, $x^2 = -y$ and $y^2 = 4x - 3$.

\item $f(x)$ is a differentiable function and $g(x)$ is a double differentiable function such that $|f(x)| \leq 1$ and
$f'(x) = g(x)$. If $f^2(0) + g^2(0) = 9$. Prove that there exists some $c \in (-3, 3)$ such that $g(c).g''(c) < 0$.

\item If
\begin{align*}
\begin{bmatrix}
4a^2 & 4a & 1 \\ 4b^2 & 4b & 1 \\ 4c^2 & 4c & 1
\end{bmatrix} \begin{bmatrix}
f(-1) \\ f(1) \\ f(2)
\end{bmatrix} = \begin{bmatrix}
3a^2 + 3a \\ 3b^2 + 3b \\ 3c^2 + 3c
\end{bmatrix}
\end{align*}
$f(x)$ is a quadrant function and its maximum value occurs at a point V. A is a point of intersection of $y = f(x)$ with x-axis and point B is such that chord AB subtends a right angle at V. Find the area enclosed by $f(x)$ and chord AB.

\item  Find the value of
\begin{align*}
5050\frac{\int_{0}^{1}(1 - x^{50})^{100}dx}{\int_{0}^{1}(1 - x^{50})^{101}dx}
\end{align*}

\item Let $f: R \to R$ be a function defined by 
\begin{align*}
f(x) = 
\left\lbrace
\begin{array}{ll}
      [x] & x \leq 2\\
      0 & x > 2\\
\end{array}
\right\rbrace
\end{align*}
where [x] is the greatest integer less than or equal to x, if 
\begin{align*}
I = \int_{-1}^{2}\frac{xf(x^2)}{2 + f(x + 1)}dx
\end{align*}
then the value of (4I - 1) is

\item Let 
\begin{align*}
F(x) = \int_{x}^{x^2 + \frac{\pi}{6}}2\cos^{2}t dt
\end{align*}
for all $x \in R$ and $f: [0, 0.5]$, if $F'(a) + 2$ is the area of the region bounded by x = 0, y = 0, y = f(x) and x = a, then f(0) is

\item If 
\begin{align*}
\alpha = \int_{0}^{1}(e^{9x + 3\tan^{-1}x})\left(\frac{12 + 9x^2}{1 + x^2}\right)dx
\end{align*}
where $\tan^{-1}x$ takes only polynomial values, then the value of $\left( log_e|1 + \alpha| - \frac{3\pi}{4}\right)$ is

\item Let $f: R \to R$ be a continuous odd function, which vanishes exactly at one point and $f(1) = \frac{1}{2}$. Suppose 
\begin{align*}
F(x) = \int_{-1}^{x}f(t)dt
\end{align*}
for all $x \in [-1, 2]$  and
\begin{align*}
G(x) = \int_{-1}^{x}t|f(f(t))|dt
\end{align*}
for all $x \in [-1, 2]$. If $\lim_{x \to 1}\frac{F(x)}{G(x)} = \frac{1}{14}$, then the value of $f(\frac{1}{2})$ is

\item The total number of distinct $x \in [0, 1]$ for which 
\begin{align*}
\int_{t^2}^{1 + t^4}dt = 2x - 1
\end{align*}
is

\item Let $f: R \to R$ be a differentiable function such that f(0)=0, $f(\frac{\pi}{2}) = 3$ and $f'(0) = 1$. If
\begin{align*}
g(x)  = \int_{x}^{\pi/2}[f'(t)\cosec t - \cot t \cosec t f(t)]dt
\end{align*}
for $x \in (0, \frac{\pi}{2}]$, then $\lim_{x \to 0}g(x)$ =

\item For positive integer n, let
\begin{align*}
y_n = \frac{1}{n}(n + 1)(n + 2).....(n + n)^{\frac{1}{n}}
\end{align*} 
For $x \in R$, let [x] be the greatest integer less than or equal to x. If $\lim_{n \to \infty}y_n = L$, then the value of $f(L)$ =

\item A farmer $F_1$ has a land in the shape of triangle with vertices at P(0, 0), Q(1, 1) and R(2, 0). From this land, a neighbouring farmer $F_2$ takes away the region which lies between the side PQ and a curve of the form $y = x^n(n > 1)$. If the area of the region taken away by the farmer $F_2$ is exactly 30 percentage of the area of $\Delta PQR$, then the value of n is...........

\clearpage

\textbf{Match the Following}

\item Match the following
\begin{table}[ht!]
\centering
\begin{tabular}{c c} 
 \textbf{Column I} & \textbf{Column II}\\ [0.5ex] 
 (A) $\int_{0}^{\pi/2}(\sin x)^{\cos x}$\\ $\left(\cos x\cot x
     - log(\sin x)^{\sin x} \right)dx$                                          &(p) 1\\ 
 (B) Area bounded by $-4y^2 = x$\\ and $x - 1 = -5y^2$                            &(q) 0\\
 (C) Cosine of angle of intersection\\ of curves
     $y = 3^{x-1}log x$\\ and $y = x^x - 1$ is                                      &(r) 6$ln2$\\                                                                     
 (D) Let $\frac{dy}{dx} = \frac{6}{x + y}$ where y(0) = 0\\
     then value of y when x + y = 6 is                                            &(s) $\frac{4}{3}$\\[1ex] 
\end{tabular}
\end{table}\\

\item Match the following
\begin{table}[ht!]
\centering
\begin{tabular}{c c} 
 \textbf{Column I} & \textbf{Column II}\\ [0.5ex] 
 (A) $\int_{-1}^{1}\frac{dx}{1 + x^2}$                      &(p) $\frac{1}{2}log\left(\frac{2}{3}\right)$\\ 
 (B) $\int_{0}^{1}\frac{dx}{\sqrt{1 - x^2}}$                &(q) $2log\left(\frac{2}{3}\right)$\\
 (C) $\int_{2}^{3}\frac{dx}{1 - x^2}$                       &(r) $\frac{\pi}{3}$\\                                                                     
 (D) $\int_{1}^{2}\frac{dx}{x\sqrt{x^2 - 1}}$               &(s) $\frac{\pi}{2}$\\[1ex] 
\end{tabular}
\end{table}\\

\item Match the following
\begin{table}[ht!]
\centering
\begin{tabular}{c c} 
 \textbf{Column I} & \textbf{Column II}\\ [0.5ex] 
 (A) The number of polynomials\\ $f(x)$ with non-negative integer\\
     coefficients of degree $\leq$ 2,\\ satisfying f(0) = 0 and\\
     $\int_{0}^{1}f(x)dx = 1$, is                                                 &(p) 8\\ 
 (B) The number of points in the\\ interval $[-\sqrt{13}, \sqrt{13}]$
     at which\\ $f(x) = \sin(x^2) + \cos(x^2)$ of\\ the maximum value is                  &(q) 2\\
 (C) $\int_{-2}^{2}\frac{3x^2}{(1 + e^x)}dx$ equals                               &(r) 4\\                                                                     
 (D) $\frac{\int_{-1/2}^{1/2}\cos 2xlog\left( \frac{1 + x}{1 - x}\right)dx}
     {\int_{0}^{1/2}\cos 2xlog\left( \frac{1 + x}{1 - x}\right)dx}$               &(s) 0\\[1ex] 
     
\textbf{codes:}
\begin{tabular}{ c c c c c}
      P & Q & R & S\\
  (a) 3 & 2 & 4 & 1\\
  (b) 2 & 3 & 4 & 1\\
  (c) 3 & 2 & 1 & 4\\
  (d) 2 & 3 & 1 & 4\\
\end{tabular}
\end{tabular}
\end{table}\\

\clearpage

\textbf{Comprehension Based Questions}

\textbf{PASSAGE-1}

Let the definite integral be defined by the fomula 
\begin{align*}
\int_{a}^{b}f(x)dx = \frac{b - a}{2}(f(a) + f(b))
\end{align*}
For more acurate results for $c \in (a, b)$ we can use
\begin{align*}
\int_{a}^{b}f(x)dx = \int_{a}^{c}f(x)dx + \int_{c}^{b}f(x)dx = F(c)
\end{align*}
so that for $c = \frac{a + b}{2}$, we get 
\begin{align*}
\int_{a}^{b}f(x)dx = \frac{b - a}{4}(f(a) + f(b) + 2f(c)).
\end{align*}

\item $\int_{0}^{\pi/2}\sin x dx$ = 
\begin{enumerate}
\item $\frac{\pi}{8}(1 + \sqrt{2})$
\item $\frac{\pi}{4}(1 + \sqrt{2})$
\item $\frac{\pi}{8\sqrt{2}}$
\item $\frac{\pi}{4\sqrt{2}}$
\end{enumerate}

\item If
\begin{align*}
\lim_{x \to a}\frac{\int_{a}^{x}f(x)dx - \left(\frac{x - a}{2}\right)(f(x) + f(a))}{(x - a)^3} = 0
\end{align*}
then $f(x)$ is of maximum degree
\begin{enumerate}
\item 4
\item 3
\item 2
\item 1
\end{enumerate}

\item If $f''(x) < 0$ for all $x \in (a, b)$ and c is a point such that $a < c < b$ and $(c, f(c))$ is the point lying on the curve for which $F(c)$ is maximum, then $f'(c)$ is equal to
\begin{enumerate}
\item $\frac{f(b) - f(a)}{b - a}$
\item $2\frac{f(b) - f(a)}{b - a}$
\item $2\frac{f(b) - f(a)}{2b - a}$
\item 0
\end{enumerate}

\textbf{PASSAGE-2}

Consider the functions defined implicity by the equation
\begin{align}
y^3 - 3y + x = 0
\end{align}
on various intervals in the real line. If $x \in (-\infty, -2) \cup (2, \infty)$, the equation implicity defines a unique real valued differentiable function $y = f(x)$. If $x \in (-2, 2)$, the equation implicity defines a unique real valued differentiable function $y = g(x)$ satisfying $g(0) = 0$.

\item If $f(-10\sqrt{2}) = 2\sqrt{2}$, then $f''(-10\sqrt{2})$ = 
\begin{enumerate}
\item $\frac{4\sqrt{2}}{7^{3}3^{2}}$
\item -$\frac{4\sqrt{2}}{7^{3}3^{2}}$
\item $\frac{4\sqrt{2}}{7^{3}3}$
\item -$\frac{4\sqrt{2}}{7^{3}3}$
\end{enumerate}

\item The area of the region bounded by the curve $y = f(x)$, the x-axis and the lines x = a and x = b, where $-\infty < a < b < -2$ is
\begin{enumerate}
\item $\int_{a}^{b}\frac{x}{3(f(x))^2 - 1}dx + bf(b) - af(a)$
\item -$\int_{a}^{b}\frac{x}{3(f(x))^2 - 1}dx + bf(b) - af(a)$
\item $\int_{a}^{b}\frac{x}{3(f(x))^2 - 1}dx - bf(b) + af(a)$
\item -$\int_{a}^{b}\frac{x}{3(f(x))^2 - 1}dx - bf(b) + af(a)$
\end{enumerate}

\item 
\begin{align*}
\int_{-1}^{1}g'(x)dx =
\end{align*} 
\begin{enumerate}
\item $2g(-1)$
\item 0
\item $-2g(1)$
\item $2g(1)$
\end{enumerate}

\textbf{PASAAGE-3}

Consider the function $f: (-\infty, \infty) \to (-\infty, \infty)$ defined by
\begin{align*}
f(x) = \frac{x^2 - ax + 1}{x^2 + ax + 1}, 0 < a < 2.
\end{align*}

\item Which of the following is True?
\begin{enumerate}
\item $(2 + a)^2 f''(1) + (2 - a)^2 f''(-1) = 0$
\item $(2 - a)^2 f''(1) - (2 + a)^2 f''(-1) = 0$
\item $f'(1)f'(-1) = (2 - a)^2$
\item $f'(1)f'(-1) = -(2 - a)^2$
\end{enumerate}

\item Which of the following is True?
\begin{enumerate}
\item $f(x)$ is decreasing on (-1, 1) and has a local minimum at x = 1
\item $f(x)$ is increasing on (-1, 1) and has a local minimum at x = 1
\item $f(x)$ is increasing on (-1, 1) but has a neither local maximum nor local minimum at x = 1
\item $f(x)$ is decreasing on (-1, 1) but has a neither local maximum nor local minimum at x = 1
\end{enumerate}

\item Let 
\begin{align*}
g(x) = \int_{0}^{e^x}\frac{f'(t)}{1 + t^2}dt
\end{align*}
Which of the following is True?
\begin{enumerate}
\item $g'(x)$ is positive on $(-\infty, 0)$ and negative on $(0, \infty)$
\item $g'(x)$ is negative on $(-\infty, 0)$ and positive on $(0, \infty)$
\item $g'(x)$ changes sign on both $(-\infty, 0)$ and $(0, \infty)$
\item $g'(x)$ does not change sign on $(-\infty, \infty)$
\end{enumerate}

\textbf{PASSAGE-4}

Consider the polynomial
\begin{align}
f(x) = 1 + 2x + 3x^2 + 4x^3
\end{align}
Let s be the sum of all distinct real roots of $f(x)$ and let $t = |s|$ 

\item The real numbers lies in the interval
\begin{enumerate}
\item $\left(-\frac{1}{4}, 0\right)$
\item $\left(-11, -\frac{3}{4}\right)$
\item $\left(-\frac{3}{4}, -\frac{1}{2}\right)$
\item $\left(0, \frac{1}{4}\right)$
\end{enumerate}

\item The area bounded by the curve $y = f(x)$ and the lines x = 0, y = 0 and x = t lies in the interval
\begin{enumerate}
\item $\left(\frac{3}{4}, 3\right)$
\item $\left(\frac{21}{64}, \frac{11}{16}\right)$
\item (9, 10)
\item $\left(0, \frac{21}{64}\right)$
\end{enumerate}

\item The function $f'(x)$ is
\begin{enumerate}
\item increasing in $\left(-t, -\frac{1}{4}\right)$ and decreasing in $\left(-\frac{1}{4}, t\right)$
\item decreasing in $\left(-t, -\frac{1}{4}\right)$ and increasing in $\left(-\frac{1}{4}, t\right)$
\item increasing in (-t, t)
\item decreasing in (-t, t)
\end{enumerate}

\textbf{PASSAGE-5}

Given that for each $a \in (0, 1)$,
\begin{align*}
\lim_{h \to 0^{+}}\int_{h}^{1 - h}t^{-a}(1 - t)^{a - 1}dt
\end{align*}
exists. Let this limit be $g(a)$. In addition, it is given that the function $g(a)$ is differentiable on (0, 1).

\item The value of $g\left(\frac{1}{2}\right)$ is
\begin{enumerate}
\item $\pi$
\item $2\pi$
\item $\frac{\pi}{2}$
\item $\frac{\pi}{4}$
\end{enumerate}

\item The value of $g'\left(\frac{1}{2}\right)$ is
\begin{enumerate}
\item $\frac{\pi}{2}$
\item $\pi$
\item $-\frac{\pi}{2}$
\item 0
\end{enumerate}

\textbf{PASSAGE-6}

Let $F: R \to R$ be a thrice differentiable function. Suppose that F(1) = 0, F(3) = -4 and $F(x) < 0$ for all $x \in \left(\frac{1}{2}, 3\right)$. Let $f(x) = xF(x)$ for all $x \in R$.

\item The correct statement(s) is(are)
\begin{enumerate}
\item $f'(1) < 0$
\item $f(2) < 0$
\item $f'(x) \neq 0$ for any $x \in (1, 3)$
\item $f'(x) = 0$ for any $x \in (1, 3)$
\end{enumerate}

\item If 
\begin{align*}
\int_{1}^{3}x^2F'(x)dx = -12
\end{align*}
and
\begin{align*}
\int_{1}^{3}x^3F''(x)dx = 40
\end{align*}
then the correct expression(s) is(are)
\begin{enumerate}
\item $9f'(3) + f'(1) - 32 = 0$
\item $\int_{1}^{3}f(x)dx = 12$
\item $9f'(3) - f'(1) + 32 = 0$
\item $\int_{1}^{3}f(x)dx = -12$
\end{enumerate}

\textbf{Integer Value Correct Type}

\item Let $f: R \to R$ be a continuous function which satisfies
\begin{align*}
f(x) = \int_{0}^{x}f(t)dt
\end{align*}
Then the value of $f(ln 5)$ is

\item For any real number x, let [x] denote the largest integer less than or equal to x. Let f be a real valued function defined on the interval [-10, 10] by
\begin{align*}
f(x) = 
\left\lbrace
\begin{array}{ll}
      x - [x] & if [x] is odd\\
      1 + [x] - x & if [x] is even\\
\end{array}
\right\rbrace
\end{align*}
Then the value of 
\begin{align*}
\frac{\pi^{2}}{10}\int_{-10}^{10}f(x)\cos \pi x dx = 
\end{align*}

\item The value of 
\begin{align*}
\int_{0}^{1}4x^3\left\lbrace\frac{d^2}{dx^2}(1 - x^2)^5\right\rbrace dx = 
\end{align*}

\item The value of the integral
\begin{align*}
\int_{0}^{1/2}\frac{1 + \sqrt{3}}{((x + 1)^2(1 - x)^6)^{1/4}}dx
\end{align*}
is

\item If 
\begin{align*}
I = \frac{2}{\pi}\int_{-\pi/4}^{\pi/4}\frac{dx}{(1 + e^{\sin x)})(2 - \cos 2x)} 
\end{align*}
then $27I^{2}$ equals........

\item The value of the integral
\begin{align*}
\int_{0}^{\pi/2}\frac{3\sqrt{\cos \theta}}{\left(\sqrt{\cos \theta} + \sqrt{\sin \theta}\right)^{5}}d\theta
\end{align*}
equals............

\textbf{Section - B}

\item $\int_{0}^{10\pi}|\sin x|dx$ is
\begin{enumerate}
\item 20
\item 8
\item 10
\item 18
\end{enumerate}

\item $I_n = \int_{0}^{\pi/4}\tan^{n}x dx$ then $\lim_{n \to \infty}n[I_n + I_{n + 2}]$ equals
\begin{enumerate}
\item 1/2
\item 1
\item $\infty$
\item 0
\end{enumerate}

\item $\int_{0}^{2}[x^2]dx$ is
\begin{enumerate}
\item $2 - \sqrt{2}$
\item $2 + \sqrt{2}$
\item $\sqrt{2} - 1$
\item $-\sqrt{2} - \sqrt{3} + 5$
\end{enumerate}

\item $\int_{-\pi}^{\pi}\frac{2x(1 + \sin x)}{1 + \cos^{2}x}dx$ is
\begin{enumerate}
\item $\frac{\pi^{2}}{4}$
\item $\pi^{2}$
\item 0
\item $\frac{\pi}{2}$
\end{enumerate}  

\item If $y = f(x)$ makes +ve intercept of 2 and 0 unit on x and y axes and encloses an area of 3/4 square unit with the axes then $\int_{0}^{2}xf'(x)dx$ is
\begin{enumerate}
\item 3/2
\item 1
\item 5/4
\item -3/4
\end{enumerate}

\item The area of bounded by the curves $y  = lnx $, $y = ln|x|$, $y = |ln x|$ and $y = |ln|x||$ is
\begin{enumerate}
\item 4 sq.units
\item 6 sq.units
\item 10 sq.units
\item None of these
\end{enumerate}

\item If $f(a + b + -x) = f(x)$, then 
\begin{align*}
\int_{a}^{b}xf(x)dx
\end{align*}
is equal to
\begin{enumerate}
\item $\frac{a + b}{2}\int_{a}^{b}f(a + b + x)dx$
\item $\frac{a + b}{2}\int_{a}^{b}f(b - x)dx$
\item $\frac{a + b}{2}\int_{a}^{b}f(x)dx$
\item $\frac{b - a}{2}\int_{a}^{b}f(x)dx$
\end{enumerate}

\item The area of the region bounded by the curves $y = |x - 1|$ and $y = 3 - |x|$ is
\begin{enumerate}
\item 6 sq.units
\item 2 sq.units
\item 3 sq.units
\item 4 sq.units
\end{enumerate}

\item Let $f(x)$ be a function satisfying $f'(x) = f(x)$ with $f(0) = 1$ and $g(x)$ be a function that satisfies $f(x) + g(x) = x^2$. Then the value of the integral
\begin{align*}
\int_{0}^{1}f(x)g(x)dx = 
\end{align*}
\begin{enumerate}
\item $e + \frac{e^2}{2} + \frac{5}{2}$
\item $e - \frac{e^2}{2} - \frac{5}{2}$
\item $e + \frac{e^2}{2} - \frac{3}{2}$
\item $e - \frac{e^2}{2} - \frac{5}{2}$
\end{enumerate}

\item The value of the integral
$I  = \int_{0}^{1}x(1 - x)^n dx$  
\begin{enumerate}
\item $\frac{1}{n + 1} + \frac{1}{n + 2}$
\item $\frac{1}{n + 1}$
\item $\frac{1}{n + 2}$
\item $\frac{1}{n + 1} - \frac{1}{n + 2}$
\end{enumerate}

\item $\lim_{n \to \infty}\sum_{r = 1}^{n}e^{\frac{r}{n}}$ = 
\begin{enumerate}
\item e + 1
\item e - 1
\item 1 - e
\item e
\end{enumerate}

\item The value of 
\begin{align*}
\int_{-2}^{3}|1 - x^2|dx = 
\end{align*}
\begin{enumerate}
\item $\frac{1}{3}$
\item $\frac{14}{3}$
\item $\frac{7}{3}$
\item $\frac{28}{3}$
\end{enumerate}

\item The value of
\begin{align*}
I = \int_{0}^{\pi/2}\frac{(\sin x + \cos x)^2}{\sqrt{1 + \sin 2x}}dx = 
\end{align*}
\begin{enumerate}
\item 3
\item 1
\item 2
\item 0
\end{enumerate}

\item If $f(x) = \frac{e^x}{1 + e^x}$,
\begin{align*}
I_1 = \int_{f(-a)}^{f(a)}xg\{x(1-x)\}dx
\end{align*}
and
\begin{align*}
I_2 = \int_{f(-a)}^{f(a)}g\{x(1-x)\}dx
\end{align*}
then the value of $\frac{I_2}{I_1}$ is
\begin{enumerate}
\item 1
\item -3
\item -1
\item 2
\end{enumerate}

\item The area of the region bounded by the curves $y = |x - 2|$, x = 1, x = 3 and the x-axis is
\begin{enumerate}
\item 4
\item 2
\item 3
\item 1
\end{enumerate}

\item If 
\begin{align*}
I_1 = \int_{0}^{1}2^{x^2}dx, I_2 = \int_{0}^{1}2^{x^3}dx
\end{align*}
\begin{align*}
I_3 = \int_{1}^{2}2^{x^2}dx, I_4 = \int_{1}^{2}2^{x^3}dx
\end{align*}
then
\begin{enumerate}
\item $I_2 > I_1$
\item $I_2 < I_1$
\item $I_3 = I_4$
\item $I_3 > I_4$
\end{enumerate}

\item The area enclosed between the curve $y = log_e(x + e)$ and the coordinate axes is
\begin{enumerate}
\item 1
\item 2
\item 3
\item 4
\end{enumerate}

\item The parabolas $y^2 = 4x$ and $x^2 = 4y$ divide the square region bounded by the lines x = 4, y = 4 and the coordinate axes. If $S_1$, $S_2$, $S_3$ are respectively the areas of these parts numbered from top to bottom; then 
$S_1:S_2:S_3$ is
\begin{enumerate}
\item 1:2:1
\item 1:2:3
\item 2:1:2
\item 1:1:1
\end{enumerate}

\item Let $f(x)$ be a non-negative continuous function such that the area bounded by the curve $y = f(x),$ x-axis and the ordinates $x = \frac{\pi}{4}$ and $x = \beta > \frac{\pi}{4}$ is $\left(\beta\sin \beta + \frac{\pi}{4}\cos \beta + \sqrt{2}\beta\right)$. Then $f(\frac{\pi}{2})$ is
\begin{enumerate}
\item $\left(\frac{\pi}{4} + \sqrt{2} -1\right)$
\item $\left(\frac{\pi}{4} - \sqrt{2} +1\right)$
\item $\left(1 - \frac{\pi}{4} - \sqrt{2}\right)$
\item $\left(1 - \frac{\pi}{4} + \sqrt{2}\right)$
\end{enumerate} 

\item The value of
\begin{align*}
\int_{-\pi}^{\pi}\frac{\cos^{2}x}{1 + a^x}dx = 
\end{align*}
\begin{enumerate}
\item $a\pi$
\item $\frac{\pi}{2}$
\item $\frac{\pi}{a}$
\item $2\pi$
\end{enumerate}

\item The value of integral
\begin{align*}
\int_{3}^{6}\frac{\sqrt{x}}{\sqrt{9 - x} + \sqrt{x}}dx = 
\end{align*}
\begin{enumerate}
\item 1/2
\item 3/2
\item 2
\item 1
\end{enumerate}

\item $\int_{0}^{\pi}xf(\sin x)dx$ = 
\begin{enumerate}
\item $\pi\int_{0}^{\pi}f(\cos x)dx$
\item $\pi\int_{0}^{\pi}f(\sin x)dx$
\item $\frac{\pi}{2}\int_{0}^{\pi/2}f(\sin x)dx$
\item $\pi\int_{0}^{\pi/2}f(\cos x)dx$
\end{enumerate}

\item 
\begin{align*}
\int_{-3\pi/2}^{-\pi/2}[(x + \pi^3) + \cos^{2}(x + 3\pi)]dx =
\end{align*} 
\begin{enumerate}
\item $\frac{\pi^{4}}{32}$
\item $\frac{\pi^{4}}{32} + \frac{\pi}{2}$
\item $\frac{\pi}{2}$
\item $\frac{\pi}{4} - 1$
\end{enumerate}

\item The value of
\begin{align*}
\int_{1}^{a}[x]f'(x)dx
\end{align*}
$a > 1$ where [x] denotes the greatest integer not exceeding x is
\begin{enumerate}
\item $af(a) - \{f(1) + f(2) +.........f([a])\}$
\item $[a]f(a) - \{f(1) + f(2) +.......f([a])\}$
\item $[a]f([a]) - \{f(1) + f(2) +........f(a)\}$
\item $af([a]) - \{f(1) + f(2) + ......f(a)\}$
\end{enumerate}

\item Let $F(x) = f(x) + f(\frac{1}{x})$, where
\begin{align*}
f(x) = \int_{l}^{t}\frac{logt}{1 + t}dt
\end{align*}
Then $F(e)$ equals
\begin{enumerate}
\item 1
\item 2
\item 1/2
\item 0
\end{enumerate}

\item The solution for x of the equation
\begin{align*}
\int_{\sqrt{2}}^{x}\frac{dt}{\sqrt{t^2 - 1}} = \frac{\pi}{2} = 
\end{align*}
\begin{enumerate}
\item $\frac{\sqrt{3}}{2}$
\item $2\sqrt{2}$
\item 2
\item None
\end{enumerate}

\item The area enclosed between the curves $y^2 = x$ and $y = |x|$ is
\begin{enumerate}
\item 1/6
\item 1/3
\item 2/3
\item 1
\end{enumerate}

\item Let
\begin{align*}
I = \int_{0}^{1}\frac{\sin x}{\sqrt{x}}dx, J = \int_{0}^{1}\frac{\cos x}{\sqrt{x}}dx 
\end{align*}
Then which one of the following is True?
\begin{enumerate}
\item $I > \frac{2}{3}$ and $J > 2$
\item $I < \frac{2}{3}$ and $J < 2$
\item $I < \frac{2}{3}$ and $J > 2$
\item $I > \frac{2}{3}$ and $J < 2$
\end{enumerate}

\item The area of the region bounded by the parabola $(y - 2)^2 = x - 1$, the tangent of the parabola at the point
(2, 3) and the x-axis is
\begin{enumerate}
\item 6
\item 9
\item 12
\item 3
\end{enumerate}

\item The area of the plane region bounded by the curves $x + 2y^2 = 0$ and $x + 3y^2 = 1$ is eqaul to
\begin{enumerate}
\item 5/3
\item 1/3
\item 2/3
\item 4/3
\end{enumerate}

\item $\int_{0}^{\pi}[\cot x]$, where [ ] denotes the greatest integer function is equal to
\begin{enumerate}
\item 1
\item -1
\item $\frac{\pi}{2}$
\item $\frac{\pi}{2}$
\end{enumerate}

\item The area of bounded by the curves $y = \cos x$ and $y = \sin x$ between the ordinates x = 0 and $x = \frac{3\pi}{2} $ is
\begin{enumerate}
\item $4\sqrt{2} + 2$
\item $4\sqrt{2} - 1$
\item $4\sqrt{2} + 1$
\item $4\sqrt{2} - 2$
\end{enumerate}

\item Let $p(x)$ be a function defined on $R$ such that $p'(x) = p'(1 - x)$, for all $x \in [0, 1]$, p(0) = 1 and p(1) = 41. Then
\begin{align*}
\int_{0}^{1}p(x)dx = 
\end{align*}
\begin{enumerate}
\item 21
\item 41
\item 42
\item $\sqrt{41}$
\end{enumerate}

\item The value of
\begin{align*}
\int_{0}^{1}\frac{8log(1 + x)}{1 + x^2}dx = 
\end{align*}
\begin{enumerate}
\item $\frac{\pi}{8}log2$
\item $\frac{\pi}{2}log2$
\item $log2$
\item $\pi log2$
\end{enumerate}

\item The area of the region enclosed by the curves y = x, x = e, $y = \frac{1}{x}$ and the positive x-axis is
\begin{enumerate}
\item 1 sq.units
\item $\frac{3}{3}$ sq.units
\item $\frac{5}{3}$ sq.units
\item $\frac{1}{2}$ sq.units
\end{enumerate}

\item The area between the parabolas $x^2 = \frac{y}{4}$ and $x^2 = 9y$ and the straight line y = 2 is
\begin{enumerate}
\item $20\sqrt{2}$
\item $\frac{10\sqrt{2}}{3}$
\item $\frac{20\sqrt{2}}{2}$
\item $10\sqrt{2}$
\end{enumerate}

\item If 
\begin{align*}
g(x)  = \int_{0}^{x}\cos 4t dt
\end{align*}
then $g(x + \pi)$ is equal to
\begin{enumerate}
\item $\frac{g(x)}{g(\pi)}$
\item $g(x) + g(\pi)$
\item $g(x) - g(\pi)$
\item $g(x).g(\pi)$
\end{enumerate}

\item 
\textbf{Statement-1:} The value of the integral
\begin{align*}
\int_{\pi/6}^{\pi/3}\frac{dx}{1 + \sqrt{\tan x}}
\end{align*}
is equal to $\pi/6$

\textbf{Statement-2:} 
\begin{align*}
\int_{a}^{b}f(x)dx = \int_{a}^{b}f(a + b -x)dx
\end{align*}
\begin{enumerate}
\item Statement-1 is true, Statement-2 is true, Statement-2 is a correct explanation for Statement-2
\item Statement-1 is true, Statement-2 is true, Statement-2 is not a correct explanation for Statement-2
\item Statement-1 is true, Statement-2 is false
\item Statement-1 is false, Statement-2 is true
\end{enumerate}

\item The area(in square units) bounded by the curves $y = \sqrt{x}$, 2y - x + 3 = 0, x-axis and lying in the first quadrant is
\begin{enumerate}
\item 9
\item 36
\item 18
\item 27/4
\end{enumerate}

\item The integral
\begin{align*}
\int_{0}^{\pi}\sqrt{1 + 4\sin^{2}\frac{x}{2} - 4\sin\frac{x}{2}}dx = 
\end{align*}
\begin{enumerate}
\item $4\sqrt{3} - 4$
\item $4\sqrt{3} - 4 - \frac{\pi}{3}$
\item $\pi - 4$
\item $\frac{2\pi}{3} - 4 - 4\sqrt{3}$
\end{enumerate}

\item The area of the region bounded by 
\begin{align*}
\{(x, y): y^2 \leq 2x, y \geq 4x - 1\} = 
\end{align*}
\begin{enumerate}
\item $\frac{15}{64}$
\item $\frac{9}{32}$
\item $\frac{7}{32}$
\item $\frac{5}{64}$
\end{enumerate}

\item The area of the region bounded by 
\begin{align*}
A = \{(x, y): x^2 + y^2 \leq 1, y^2 \leq 1 - x\} = 
\end{align*} 
\begin{enumerate}
\item $\frac{\pi}{2} - \frac{2}{3}$
\item $\frac{\pi}{2} + \frac{2}{3}$
\item $\frac{\pi}{2} + \frac{4}{3}$
\item $\frac{\pi}{2} - \frac{4}{3}$
\end{enumerate}

\item The integral
\begin{align*}
\int_{2}^{4}\frac{log x^2}{log x^2 + log(36 - 12x + x^2)}dx
\end{align*}
is equal to
\begin{enumerate}
\item 1
\item 6
\item 2
\item 4
\end{enumerate}

\item The area(in square units) of the region
\begin{align*}
\{(x, y): y^2 \geq 2x, x^2 + y^2 \leq 4x, x \geq 0, y \geq 0\} = 
\end{align*}
\begin{enumerate}
\item $\pi - \frac{4\sqrt{2}}{•3}$
\item $\frac{\pi}{2} - \frac{2\sqrt{2}}{3}$
\item $\pi - \frac{4}{3}$
\item $\pi - \frac{8}{3}$
\end{enumerate}

\item The area(in square units) of the region
\begin{align*}
\{(x, y): x \geq 0, x + y \leq 3, x^2 \leq 4y, y \geq 1 + \sqrt{x}\} = 
\end{align*}
\begin{enumerate}
\item 5/2
\item 59/12
\item 3/2
\item 7/3
\end{enumerate}

\item The integral
\begin{align*}
\int_{\pi/4}^{3\pi/4}\frac{dx}{1 + \cos x} = 
\end{align*}
\begin{enumerate}
\item -1
\item -2
\item 2
\item 4
\end{enumerate}

\item Let $g(x) = \cos^{2}x$, $f(x) = \sqrt{x}$ and $\alpha, \beta (\alpha < \beta)$ be the roots of the quadratic equation
\begin{align}
18x^2 - 9\pi x + \pi^2 = 0
\end{align} 
Then the area(in sq.units) bounded by the curve $y = (gof)(x)$ and he lines $x = \alpha$, $x = \beta$ and y = 0 is
\begin{enumerate}
\item $\frac{1}{2}(\sqrt{3} + 1)$
\item $\frac{1}{2}(\sqrt{3} - \sqrt{2})$
\item $\frac{1}{2}(\sqrt{2} - 1)$
\item $\frac{1}{2}(\sqrt{3} - 1)$
\end{enumerate}

\item The value of
\begin{align*}
\int_{-\pi/2}^{\pi/2}\frac{\sin^{2}x}{1 + 2^{x}}dx = 
\end{align*}
\begin{enumerate}
\item $\frac{\pi}{2}$
\item $4\pi$
\item $\frac{\pi}{4}$
\item $\frac{\pi}{8}$
\end{enumerate}

\item The value of
\begin{align*}
\int_{0}^{\pi}|\cos x|dx = 
\end{align*}
\begin{enumerate}
\item 0
\item 4/3
\item 2/3
\item -4/3
\end{enumerate}

\item The area(in sq.units) bounded by the parabola $y = x^2 - 1$, the tangent at the point (2, 3) to it and the y-axis is
\begin{enumerate}
\item 8/3
\item 32/3
\item 56/3
\item 14/3
\end{enumerate}

\item The value of
\begin{align*}
\int_{0}^{\pi/2}\frac{\sin^{3}x}{\sin x + \cos x}dx = 
\end{align*}
\begin{enumerate}
\item $\frac{\pi - 2}{8}$
\item $\frac{\pi - 1}{4}$
\item $\frac{\pi - 2}{4}$
\item $\frac{\pi - 1}{2}$
\end{enumerate}

\item The area(in sq.units) of the region
\begin{align*}
A = \{(x, y): x^2 \leq y \leq x + 2\} = 
\end{align*}
\begin{enumerate}
\item 10/3
\item 9/2
\item 31/6
\item 13/6
\end{enumerate}









